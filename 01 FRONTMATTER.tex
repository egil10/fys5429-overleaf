% 01 FRONTMATTER.tex
\twocolumn[
\begin{@twocolumnfalse}

% -------------------- TITLE BLOCK --------------------
\begin{center}
{\Huge\bfseries PINNs for Option Pricing \par}
\vspace{0.2cm}
{\Large From Black-Scholes to Stochastic Volatility Calibration \par}
\vspace{0.2cm}
{\large FYS5429: Advanced Machine Learning and Data Analysis for the Physical Sciences \par}
\vspace{0.2cm}
{\large\bfseries Project 1 \par}
\vspace{0.2cm}
\begin{minipage}{0.48\linewidth}
\raggedright
{\large Egil Furnes \par}
egilsf@uio.no \par
\href{https://github.com/egil10/fys5429}{\texttt{github.com/egil10/fys5429}}
\end{minipage}
\hfill
\begin{minipage}{0.48\linewidth}
\raggedleft
University of Oslo \par
Department of Physics \par
\today
\end{minipage}
\end{center}

% -------------------- ABSTRACT --------------------
\vspace{1em}
\begin{center}
{\large\bfseries Abstract}
\end{center}
\vspace{1em}
\noindent

Financial markets exhibit pronounced non-stationarities driven by structural breaks, macroeconomic shocks, and evolving investor behavior. This project investigates the use of machine learning methods for identifying and modeling regime changes in financial time series. We consider supervised and unsupervised approaches to regime detection, including clustering-based methods and probabilistic state models, and evaluate their ability to capture shifts in volatility, return dynamics, and cross-asset dependencies. Model performance is assessed using historical market data under realistic out-of-sample settings. The results highlight both the potential and limitations of machine-learning-based regime modeling in quantitative finance, with implications for risk management, portfolio allocation, and adaptive trading strategies.

\vspace{2em}

\end{@twocolumnfalse}
]
