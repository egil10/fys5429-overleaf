\section{Methods}\label{section:methods}

This is the methods section, lorem ipsum.

Modern quantitative finance sits at the intersection of stochastic modeling, numerical methods, and machine learning. Classical stochastic volatility modeling builds on multiscale diffusion techniques, where volatility evolves on multiple time scales, as described by \textcite{fouque2011multiscale}. Analytical approaches to option pricing under stochastic dynamics are further developed in tractable model frameworks such as those presented in \textcite{gulisashvili2012analytically}, while probabilistic interpretations of derivative pricing link option values to risk-neutral expectations \parencite{optionsprobabilities}. 

In parallel, machine learning methods have become increasingly relevant for solving high-dimensional approximation problems arising from partial differential equations and financial data modeling. Foundational deep learning theory and architectures are presented in \textcite{goodfellow2016deep}, while practical implementations using modern software frameworks are discussed in \textcite{raschka2022ml}. Reinforcement learning methods provide an additional computational paradigm for sequential decision-making and dynamic optimization problems in finance \parencite{sutton2018rl}. Finally, time-change techniques offer an alternative stochastic modeling framework that connects diffusion processes with subordinated dynamics in financial markets, as discussed in \textcite{swishchuk2008time}. Together, these perspectives motivate the use of physics-informed machine learning methods for solving option pricing problems governed by stochastic differential equations and partial differential equations.
